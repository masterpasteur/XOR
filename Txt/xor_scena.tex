\documentclass{article}

\input ../sketch.sty

\def\xor{\textsc{xor}}

\title{XOR : scénario}
\author{GdJ EEL 2012}
\date{10 11 12}

\begin{document}

\maketitle

	\subsection{Introduction}

		\begin{itemize}
		\item \histoire{} Les recruteurs s'intéresseront à des qualités 
		opposées : foi / compétence.
		\item \mes{} L'un et l'autre parlent à tour de rôle sur le même
		sujet mais avec des points de vues différents, souvent opposés.
		\item \mes{}/\histoire{} Henry introduit le sketch par voix-off : 
		\ofg{Voici comment ça s'est passé.}
		\item \histoire{} Henry annonce la nouvelle à sa mère et sait qu'il doit
		faire un choix : entrer dans le monde du travail
		et gagner sa vie \xor{} partir en voyage missionnaire pour changer 
		de vie et servir Dieu.
		\item \humour{} : Henry répond à sa mère : \ofg{Non, pas possible de 
		faire les deux.}
		\end{itemize}
	
	\subsection{Hésitation et décision}
	
		\begin{itemize}
		\item \mes{} Garder l'opposition des 2 voies et la marquer
		spatialement sur scène, i.e. droite : association / gauche : entreprise.
		\item \histoire{} Henry reste silencieux mais les autres donnent leur
		avis. Henry ne prend la décision qu'à la fin. Cela permet
		aux spectateurs de s'identifier à lui.
		\item Personnages et interactions :
			\begin{itemize}
			\item Un ami : \histoire{} donne des conseils de bon sens.
			\item Sa mère : \humour{} pour le travail : \ofg{Qui va payer les
			études ?}
			\item La fille dont il est amoureux : \humour{}
			Henry est paralysé / Elle a un discours bateau pour le voyage
			missionnaire.
			\item Son mentor : \drama{} Il fait tomber son masque
			\item Ses recruteurs : le chef de l'entreprise et un responsable
			de l'association : \drama{}/\histoire{} : Discussion sérieuse,
			retour sur le passé.
			\end{itemize}
		\item \histoire{} Décision : après un brouhaha, il demande
		le silence. Tous les autres personnages sortent (de sa tête).
		Et Henry prend calmement et sagement sa décision, seul.
		\end{itemize}

	\subsection{Retour au présent}
	
		\begin{itemize}
		\item \humour{} Un collègue est rivé sur son iPhone. Il demande
		s'il est sorti avec la fille.
		\item \histoire{}/\evang{} L'autre collègue pourrait se demander :
		\ofg{Pourquoi ce voyage missionnaire ?}
		\item \evang{}/\drama{} La transition vers l'annonce de l'Evangile doit
		être réaliste.
			\begin{itemize}
			\item \histoire{} Henry s'inspire de la \emph{cathédrale} et du
			\emph{bazar}, du monde \ofg{privateur} et du monde \ofg{libre} pour
			annoncer l'évangile. Note : la comparaison est imparfaite mais
			pertinente.
			\item \mes{} Il a du mal au début mais prend son envol.
			\item \mes{}/\histoire{} Il n'a pas le temps de terminer qu'ils 
			doivent reprendre le boulot. Les collègues veulent en parler une
			prochaine fois.
			\end{itemize}
		\end{itemize}

\end{document}