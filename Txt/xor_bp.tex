\documentclass{article}

\input ../sketch.sty

\title{XOR : brouillon}
\author{sc}
\date{27 10 12}

\begin{document}

\maketitle
    
    \begin{abstract}
        \begin{itemize}
        \item Cette année, je devais choisir entre un stage avec les GBU
        (Groupes Bibliques Universitaires) et un doctorat de physique.
        \item Ce sketch s'inspire de ce choix difficile que j'ai dû faire.
        \item Son message est simple : d'abord évangéliser nos proches,
        nos amis et nos collègues.
        \end{itemize}
    \end{abstract}
    
\section{Préparation}

    \subsection{Synopsis}

        Le pitch
        Henry est fraîchement diplômé en informatique.
        Il hésite entre un premier emploi dans une entreprise
        et un voyage missionnaire avec une association
        chrétienne.
        
        \begin{itemize}
        \item Reprise des éléments biographiques
        \item Suspense : quel choix ?
        \item Voyage dans le temps :
            Histoireception
            Henry raconte à ses collègues son parcours personnel
            et les motivations de son choix.
        \end{itemize}
        
    \subsection{Personnages}
    
        \begin{itemize}
        \item \perso{Henry} : un geek asocial
        \item \perso{2 recruteurs} : 
            \begin{itemize}
            \item association : encourageant 
            \item entreprise : cassant
            \end{itemize}
        \item \perso{3 amis} qui le conseillent :
            \begin{itemize}
            \item son meilleur ami et mentor
            \item la fille dont il est amoureux
            \item sa s\oe ur possessive
            \end{itemize}
        \item \perso{2 collègues} de travail :
            \begin{itemize}
            \item l'un qui s'en fout
            \item l'autre qui s'intéresse
            \end{itemize}
        \end{itemize}
    
    \subsection{Environnement}
    
        L'idée
        Le monde dans lequel nous vivons,
        les gens que nous côtoyons.
        
        \begin{itemize}
        \item En région parisienne
        \item Monde urbain, \ofg{élitiste}, e.g. La Défense
        \item Athées, agnostiques
        \end{itemize}
        
    \subsection{Mise en scène}
    
        Jeu d'oppositions
            \begin{itemize}
            \item Imaginaire / Réaliste
            \item Symbolique / Flou
            \item Flashback / Live
            \end{itemize}
    
\section{Scénario}

    Structure
        \begin{enumerate}
        \item Introduction : recrutements en parallèle
        \item Hésitation et décision : tensions, convictions et doutes
        \item Retour au présent : occasion d'annoncer la Bonne Nouvelle
        à ses collègues
        \end{enumerate}

    \subsection{Introduction}
    
        \begin{itemize}
        \item Les recruteurs s'intéresseront à des qualités opposées :
        foi / compétence.
        \end{itemize}
    
    \subsection{Hésitation et décision}    
    
        \begin{itemize}
        \item Garder l'opposition des 2 voies et la marquer
        spatialement sur scène (e.g. droite : association / 
        gauche : entreprise).
        \item Henry reste silencieux mais les autres donnent leur
        avis. Henry ne prend la décision qu'à la fin. Cela permet
        aux spectateurs de s'identifier à lui.
        \end{itemize}
    
    \subsection{Retour au présent}
    
        \begin{itemize}
        \item La transition vers l'annonce de l'Evangile doit être
        réaliste.
        \item Un collègue pourrait se demander :
        \ofg{Pourquoi ce voyage missionnaire ?}
        \end{itemize}    
    
\section{Pour la prochaine fois}

    To do list : 10/11/12
        \begin{itemize}
        \item Scénario : l'écrire en plus détaillé.
        \item Dialogues : à rédiger.
        \end{itemize}
	
\end{document}