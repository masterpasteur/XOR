\documentclass{article}

\input ../sketch.sty

\def\henry{Henry}
\def\assoResp{Responsable de l'association}
\def\compDir{Directeur de l'entreprise}
\def\maman{Maman}
\def\ami{Un ami}
\def\fille{La belle fille}
\def\mentor{Le mentor}
\def\collIphone{Collègue out}
\def\collIntr{Collègue in}

\title{XOR : dialogues}
\author{GdJ EEL 2012}
\date{10 11 12}

\begin{document}

\maketitle

	\subsection{Introduction}
	
		\subsubsection{Recrutement}
	
		\dida{Les 2 recruteurs entrent, l'un à droite, l'autre à gauche
		de la scène. Puis ils s'immobilisent.}
		
		\dida{Voix-off d'Henry}
		\begin{itemize}
		\repItem{\henry}{OK, très bien. Voici comment tout a commencé.}
		\end{itemize}
		
		\dida{Les recruteurs prennent la parole tour à tour mais
		ils ne se voient pas. Ils parlent à leurs collègues, invisibles.}
		
		\dida{On commence dans l'association chrétienne.}
		\begin{itemize}
		\repItem{\assoResp}{Ensuite, c'est au tour d'Henry. Qu'est-ce que
		vous en pensez ? Yohann ?}
		\repItem{Yohann}{\ldots} \dida{Le responsable acquiesce de la tête.}
		\repItem{\assoResp}{Oui, tu as raison. C'est quelqu'un de bien.
			Sa foi est ferme, il n'a y aucun doute. 
			\dida{Silence}
			Par contre, j'ai un doute sur, comment dire\ldots} 
		\repItem{Marion}{\ldots}
		\repItem{\assoResp}{Oui, c'est ça, Marion. J'ai peur qu'il ne soit
			pas assez proche des autres membres du groupe.}
		\end{itemize}
		
		\dida{On bascule dans l'entreprise.}
		\begin{itemize}
		\repItem{\compDir}{Le prochain candidat est : Henry \nom{Charles}.
			La parole est à vous.} \dida{Il regarde et écoute ses collaborateurs.}
		\repItem{Collaborateurs}{\ldots}
		\repItem{\compDir}{Je suis d'accord avec vous : M. \nom{Charles} est 
			quelqu'un de compétent. Mais il aura énormément de mal à
			s'intégrer dans l'équipe.}
		\repItem{un collaborateur}{\ldots}
		\repItem{\compDir}{Non, je ne juge pas sur l'apparence. J'ai pu voir
			ce qu'il valait pendant son stage ici, dans notre entreprise.}
		\end{itemize}
		
		\dida{On retourne à l'association.}
		\begin{itemize}
		\repItem{\assoResp}{Bon, les amis, on doit prendre une décision.
			Est-ce que quelqu'un est contre ?
			Oui, Jonathan ? Qu'est-ce qu'il y a ?}
		\repItem{Jonathan}{\ldots}
		\repItem{\assoResp}{Oui, tu as raison de soulever ce point.
			\dida{Il s'adresse à tous.} En plus de cette candidature pour le
			voyage missionnaire au Mali, Henry a aussi passé un entretient
			pour travailler dans une entreprise informatique.}
		\repItem{Paul}{\ldots}
		\repItem{\assoResp}{Je suis d'accord avec toi, Paul. C'est assez
			étrange d'envisager ses deux directions plutôt opposées. Mais 
			je pense qu'il cherche	simplement la volonté de Dieu.}
		\end{itemize}
		\dida{Le responsable sort de la scène.} % brutal ?
		
		\dida{Dernier switch dans l'entreprise.}
		\begin{itemize}
		\repItem{\compDir}{Chers collègues, nous devons nous décider
			rapidement. Nous avons d'autres candidatures à analyser.}
		\item \dida{Les collaborateurs discutent. Le directeur les regardent
			et écoutent.}
		\repItem{\compDir}{Excusez-moi, mais je crois que nous n'avons plus 
			le temps. Je vous propose de revenir sur ce dossier à la fin.}
		\end{itemize}
		\dida{Le directeur sort de la scène.}% brutal ?
		
		\subsubsection{Le choix}
			
			\dida{Henry entre en scène avec son kit main libre et 2 lettres.}
			\begin{itemize}
			\repItem{\henry}{Allô maman ? C'est moi, Henry, ton fils.}
			\repItem{\maman}{\ldots}
			\repItem{\henry}{Oui, maman. J'ai reçu les réponses, sous formes
				de lettres imprimées, dans ma boîte aux lettres.}
			\repItem{\maman}{\ldots}
			\repItem{\henry}{Ce qu'elles disent ? 
					\dida{Il tend une première lettre d'un côté.}
					La première lettre me dit que ma candidature a été retenue
					pour le voyage missionnaire. 
					\dida{Il tend l'autre lettre.}
					Et la deuxième lettre m'informe que\ldots l'entreprise
					a accepté de m'embaucher.}
			\item \dida{Sa mère crie de joie. Henry a mal aux oreilles.}
			\repItem{\henry}{Oui, maman, qu'est-ce qu'il y a?}
			\repItem{\maman}{\ldots}
			\repItem{\henry}{Non, maman, je ne peux faire les deux à la fois.}
			\repItem{\maman}{\ldots}
			\repItem{\henry}{Pourquoi ? Bah, parce que je ne peux être au Mali
				et à Paris en même temps.}
			\repItem{\maman}{\ldots}
			\repItem{\henry}{Quelle décision je vais prendre ? Je ne sais pas.
				J'ai une semaine pour me décider.}
			\end{itemize}
			\dida{Henry sort de la scène.}
	
	\subsection{Discussion}
	
		\subsubsection{Hésitations}
		
		\dida{Henry entre en scène. Il est pensif.}
		\begin{itemize}
		\repItem{\henry{} \dida{Voix-off}}{Après, j'ai parlé avec des amis pour
			leur demander conseil.}
		\end{itemize}
		
		\dida{Un ami arrive}
		\begin{itemize}
		\repItem{\ami}{Salut Henry ! \c Ca va ?}
		\repItem{\henry}{Je pense que je vais bien. Mais\ldots}
		\repItem{\ami}{Toujours en pleine réflexion ?}
		\item \dida{Henry acquiesce.}
		\repItem{\ami}{Quand tu fais un choix important, tu dois penser
			à trois choses.}
		\repItem{\henry}{Quelles sont ces trois choses ?}
		\repItem{\ami}{D'abord, est-ce que ton choix est \emph{moralement}
			bon ? Dans les 2 cas, je pense qu'il n'y a pas de problème.}
		\repItem{\henry}{C'est vrai. Ensuite ?}
		\repItem{\ami}{Ensuite, tu dois te demander quel sera le choix
			le plus utile. Est-ce tu pourras mieux servir Dieu au Mali
			en aidant les gens ou dans une entreprise, en parlant de Dieu
			à tes collègues ?}
		\repItem{\henry}{On doit parler de Dieu à ses collègues ?}
		\repItem{\ami}{Oui, c'est évident, non ?}
		\repItem{\henry}{J'aurais jamais le courage de faire ça !}
		\repItem{\ami}{Ne t'inquiète pas. Dieu ne te lâchera pas.}
		\repItem{\henry}{Et quel est ton dernier conseil ?}
		\repItem{\ami}{Ah oui ! Et enfin, si c'est OK pour ces 2 quesitons,
			tu es libre de prendre la décision qui te plaît.}
		\repItem{\henry}{Ouais, ça ne m'a pas beaucoup avancé.}
		\repItem{\ami}{Bon courage Henry. Tiens-moi au courant !}
		\end{itemize}
		\dida{L'ami sort de la scène.}
		
		\dida{Henry continue a marché sur scène et à réfléchir. 
		Quand soudain\ldots}
		
		\begin{itemize}
		\repItem{\fille}{Salut !}
		\dida{Henry est totalement paralysé devant la fille.}
		\repItem{\fille}{On s'est déjà vu à un rassemblement de jeunes.
			Comment tu t'appelles, déjà ?}
		\repItem{\henry}{\ldots}
		\dida{La fille sourit.}
		\repItem{\fille}{On m'a dit que tu réfléchissait à ton avenir.}
		\item \dida{Henry acquiesce de la tête.}
		\repItem{\fille}{Je ne connais pas les détails. Mais je te conseille
			de remettre tous tes fardeaux à Dieu. Car lui seul sait ce qui est 
			bon pour toi.}
		\repItem{\henry}{\ldots}
		\repItem{\fille}{Bon, je file. Je vais au ciné avec des amis.
			Salut, à la prochaine !}
		\repItem{\henry}{\ldots}
		\end{itemize}
		\dida{La fille sort de la scène.}
		
		\dida{Henry essaie de se reprendre. Son mentor arrive.}
		\begin{itemize}
		\repItem{\mentor}{Salut Henry ! Comment vas-tu ?}
		\repItem{\henry}{Je suis\ldots perturbé.}
		\repItem{\mentor}{J'ai vu une fille passée. C'est ta copine ?}
		\repItem{\henry}{Je n'ai pas de copine.}
		\item \ldots
		\end{itemize}
		
		\subsubsection{Décision}
	
	\subsection{Retour au présent}

\end{document}