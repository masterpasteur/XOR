\documentclass{article}

\input sketch.sty

%Personnages :
\def\henry{Henry}
\def\assoResp{Responsable de l'association}
\def\compDir{Directeur de l'entreprise}
\def\maman{Maman}
\def\ami{Ami}
\def\fille{Belle fille}
\def\mentor{Mentor}
\def\collOut{Collègue out}
\def\collIn{Collègue in}

\title[XOR]{Sketch de la fête de Noël 2012 :\\ XOR (eXclusive OR)}
\author[EEL.GdJ]{Groupe de jeunes de l'Eglise Evangélique du Lac}
\date[22/12/12]{Fête de Noël 2012 : samedi 22 décembre}

\begin{document}

\maketitle

\textit{A Nicolas}

\begin{abstract}
Cette année, je devais choisir entre un stage avec les GBU
(Groupes Bibliques Universitaires) et un doctorat de physique.
Ce sketch s'inspire de ce choix difficile.
Son message est simple : d'abord évangéliser nos proches,
nos amis et nos collègues.
\end{abstract}

\setcounter{tocdepth}{2}
\tableofcontents

%\input Sections/prep.tex
	
%\input Sections/scena.tex

\section{Dialogues}

	\input Sections/dialogues_intro.tex
	\input Sections/dialogues_disc.tex
	\input Sections/dialogues_evang.tex

%\input Sections/toDo.tex

%\input Sections/insp.tex

\end{document}