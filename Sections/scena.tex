\section{Scénario}

\begin{frame}
\sectionFrame{}
\end{frame}
	
\begin{frame}
	\begin{block}{Structure}
		\begin{enumerate}
		\item Introduction : recrutements en parallèle
		\item Hésitation et décision : tensions, convictions et doutes
		\item Retour au présent : occasion d'annoncer la Bonne Nouvelle
		à ses collègues
		\end{enumerate}
	\end{block}
\end{frame}

	\subsection{Introduction}
	
	\begin{frame}[allowframebreaks]
	\frametitle{\insertsubsection}

		\begin{itemize}
		\item \histoire{} Les recruteurs s'intéressent à des qualités 
		différentes : foi / compétence.
		\item \mes{} L'un et l'autre parlent à tour de rôle sur le même
		sujet mais avec des points de vues différents, souvent opposés.
		\item \mes{}/\histoire{} Henry introduit le sketch par voix-off : 
		\ofg{Voici comment ça s'est passé.}
		\item \histoire{} Henry annonce la nouvelle à sa mère et sait qu'il doit
		faire un choix : entrer dans le monde du travail
		et gagner sa vie \emph{xor} partir en voyage missionnaire pour 
		changer de vie et servir Dieu.
		\item \humour{} : Henry répond à sa mère : \ofg{Non, pas possible de 
		faire les deux.}
		\end{itemize}
	
	\end{frame}
	
	\subsection{Hésitation et décision}
	
	\begin{frame}[allowframebreaks]
	\frametitle{\insertsubsection}
	
		\begin{block}{Opposition}
		\mes{} Garder l'opposition des 2 voies et la marquer
		spatialement sur scène, i.e. droite : association / gauche : entreprise.
		\end{block}
	
		\begin{itemize}
		\item \histoire{} Henry reste silencieux mais les autres donnent leur
		avis. Henry ne prend la décision qu'à la fin. Cela permet
		aux spectateurs de s'identifier à lui.
		\item Personnages et interactions :
			\begin{itemize}
			\item \histoire{} Un ami : donne des conseils de bon sens.
			\item \humour{} Sa mère : pour le travail : \ofg{Qui va payer les
			études ?}
			\item \humour{} La fille dont il est amoureux : Henry est paralysé /
			Elle a un discours assez vague mais vrai sur le Dieu omniscient.
			\item \drama{} Son mentor : Il fait tomber son masque.
			Henry est content de revoir son mentor. Il peut enfin
			déballer tout ce qu'il a dans le coeur. Mais son coach
			n'est pas du même avis. Il en a marre d'entendre Henry
			répéter les mêmes choses. Il lui montre que sa vision
			est biaisée (e.g. La pitié pour les petits Africains
			\textrightarrow son mépris en fait). Cela met Henry très 
			en colère. Et ils se séparent brutalement.
			\end{itemize}
		\end{itemize}
		
		\begin{itemize}
		\item Chacun des protagonistes passés revient avec ses
		répliques. Cela embrouille énormément Henry.
		Et il les chasse de la scène.
			\begin{block}{\item Décision}
			\histoire{}/\drama{} Après un brouhaha, il demande
			le silence. Tous les autres personnages sortent (de sa tête).
			Et Henry prend calmement et sagement sa décision, seul.
			\end{block}
		\item Henry se sent libéré. Il se réconcilie avec mentor.
		\end{itemize}
	
	\end{frame}
	
	\subsection{Retour au présent}
	
	\begin{frame}[allowframebreaks]
	\frametitle{\insertsubsection}
	
		\begin{itemize}
		\item \humour{} Une collègue est rivée sur son smartphone. Elle demande
		s'il est sorti avec la fille.
		\item \histoire{}/\evang{} L'autre collègue pourrait se demander :
		\ofg{Pourquoi ce voyage missionnaire ?}
		\item \evang{} La transition vers l'annonce de l'Evangile doit
			être réaliste. Quel angle d'attaque choisir ?
			\begin{itemize}
			\item Henry pourrait s'inspirait de la \emph{cathédrale} et du
			\emph{bazar}, du monde \ofg{privateur} et du monde \ofg{libre} pour
			annoncer l'évangile. La comparaison est imparfaite mais
			peut-être pertinente. Je dois faire des recherches pour vérifier.
			\item Sinon, il pourrait parler de la sécurité informatique, e.g.,
			\emph{virus versus antivirus}.
			\item Ou il pourrait aborder le problème de notre (ma ?) dualité 
			(schizophrénie ?) : devant un ordinateur et dans la vie réelle.
			\item Autre possibilité :
			Quand ses collègues lui demandent ses motivations,
			il commence par sa compassions pour les Africains.
			Mais en fait, c'est lui qui était dans une situation
			critique [Question: L'a-t-il compris pendant son silence ?
			Avant ?]. Il a honte de parler de Dieu. Mais ses collègues
			ne sont pas gênés.
			\item[+] \mes{}/\drama{} Il a du mal au début mais prend son envol.
			\end{itemize}
		\item \mes{}/\histoire{} Il n'a pas le temps de terminer qu'ils 
		doivent reprendre le boulot. Les collègues veulent en parler une
		prochaine fois.
		\end{itemize}
	
	\end{frame}