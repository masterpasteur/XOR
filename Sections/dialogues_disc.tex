\subsection{Discussions}

	\subsubsection{Hésitations}
	
	\begin{itemize}
	\repItem{\henry{} \dida{Voix-off}}{\off{C'était un choix difficile.
		J'ai donc demandé conseil à des amis.}}
	\end{itemize}
	
	\dida{Henry entre en scène. Il est pensif. Un ami arrive.}
	\begin{itemize}
	\repItem{\ami}{Salut Henry ! \c Ca va ?}
	\repItem{\henry}{Je pense que je vais bien. Mais\ldots}
	\repItem{\ami}{Toujours en pleine réflexion ?}
	\item \dida{\cast{\henry} acquiesce.}
	\repItem{\ami}{Quand tu fais un choix important, tu dois penser
		à trois choses.}
	\repItem{\henry}{Quelles sont-elles ?}
	\repItem{\ami}{D'abord, est-ce que ton choix est \emph{moralement}
		bon ? Dans les 2 cas, je pense qu'il n'y a pas de problème.}
	\repItem{\henry}{Mais c'est mal de gagner de l'argent, beaucoup
		d'argent !}
	\repItem{\ami}{Qu'est-ce que tu racontes ? Au contraire, tu pourras
		en donner à ceux qui en ont besoin. A moi, par exemple !}
	\item \dida{\cast{\henry} acquiesce de la tête.}
	\repItem{\ami}{Ensuite, tu dois te demander quel sera le choix
		le plus utile.}
	\repItem{\henry}{C'est-à-dire ?}
	\repItem{\ami}{Est-ce que tu pourras mieux servir Dieu au Mali,
		en aidant les gens ou dans une entreprise, en parlant de Dieu
		à tes collègues ?}
	\repItem{\henry}{On doit parler de Dieu à ses collègues ?}
	\repItem{\ami}{Oui, c'est évident, non ?}
	\repItem{\henry}{Je n'aurais jamais le courage de faire ça !}
	\repItem{\ami}{Ne t'inquiète pas. Dieu sera toujours avec toi.}
	\repItem{\henry}{\dida{Un peu agacé.} Et quel est ton dernier conseil ?}
	\repItem{\ami}{Ah oui ! Et enfin, si c'est OK pour ces 2 questions,
		tu es libre de prendre la décision qui te plaît.}
	\repItem{\henry}{Merci. Mais ça ne m'a pas beaucoup avancé.}
	\repItem{\ami}{Bon courage Henry. Tiens-moi au courant !}
	\end{itemize}
	\dida{L'ami sort de la scène.}
	
	\dida{Henry continue à marcher sur scène et à réfléchir, 
	quand soudain\ldots}	
	\begin{itemize}
	\repItem{\fille}{Salut !}
	\repItem{\henry}{\dida{Il est totalement paralysé devant la fille, 
		bouche bée.}}
	\repItem{\fille}{On s'est déjà vu à un rassemblement de jeunes.
		Comment tu t'appelles, déjà ?}
	\repItem{\henry}{\ldots}
	\repItem{\fille}{\dida{La fille sourit.} On m'a dit que tu réfléchissais
		à ton avenir.}
	\repItem{\henry}{\dida{Il acquiesce de la tête.}}
	\repItem{\fille}{Je ne connais pas les détails. Mais je te conseille
		de remettre tous tes fardeaux à Dieu. Car lui seul sait ce qui est 
		bon pour toi.}
	\repItem{\henry}{\ldots}
	\repItem{\fille}{Bon, je file. Je vais au ciné avec des amis.
		Salut, à la prochaine !}
	\repItem{\henry}{\ldots}
	\end{itemize}
	\dida{La fille sort de la scène.}
	
	\dida{Henry essaie de se reprendre. Son mentor arrive.}
	\begin{itemize}
	\repItem{\mentor}{Salut Henry ! Comment ça va ?}
	\repItem{\henry}{Je suis\ldots perturbé.}
	\repItem{\mentor}{J'ai vu une fille passée. C'est ta copine ?}
	\repItem{\henry}{Je n'ai pas de copine.}
	\repItem{\mentor}{C'est pas grave ! T'as un travail. T'en as même deux,
		c'est ça ?}
	\repItem{\henry}{Oui, malheureusement.}
	\repItem{\mentor}{T'es jamais content, en fait?}
	\repItem{\henry}{Non, c'est faux. Par exemple, je suis très content
		de te voir. Comme ça on pourra discuter comme la dernière fois.}
	\repItem{\mentor}{\c Ca ressemblait plus à un monologue qu'à
		une discussion\ldots}
	\repItem{\henry}{Tu ne veux plus m'écouter ?}
	\repItem{\mentor}{Je suis prêt à t'écouter, si c'est utile.}
	\repItem{\henry}{Comment tu peux dire ça ?}
	\repItem{\mentor}{Henry, on n'a plus le temps de discuter.
		Tu dois prendre ta décision ce soir.}
	\repItem{\henry}{Dis plutôt que tu ne veux pas discuter avec moi
		et partir.}
	\repItem{\mentor}{Henry, je vais te laisser réfléchir et\ldots}
	\repItem{\henry}{Oui. Va t'en, s'il te plaît.}
	\repItem{\mentor}{\dida{quitte la scène en silence.}}
	\repItem{\henry}{\dida{regrette déjà ses paroles et ses gestes.}}
	\end{itemize}

	\subsubsection{Décision}
	
	\dida{Henry reste sur scène. Les personnages vont rentrer un à un,
	dans l'ordre inverse, avec les mêmes répliques. Ils tournent autour
	de lui comme pour le dévorer.}
	\begin{itemize}
	\repItem{\fille}{Salut ! On s'est déjà vu à un rassemblement de jeunes.
		Comment tu t'appelles, déjà ?}
	\repItem{\ami}{Salut Henry ! \c Ca va ? Toujours en pleine réflexion ?}
	\repItem{\henry}{Allô maman ? C'est moi, Henry. \dida{Il se demande
		ce qu'il se passe.}}
	\repItem{\compDir}{Le prochain candidat est : Henry \nom{Charles}.}
	\repItem{\assoResp}{Ensuite, c'est au tour d'Henry.}
	\repItem{\henry}{\textsc{Stop !!!}}
	\item \dida{Silence.}
	\repItem{\henry}{S'il vous plaît, sortez. Sortez de ma tête.
		Laissez-moi réfléchir.}
	\end{itemize}
	\dida{Tous les autres personnages sortent.}
	
	\dida{Henry pense. Il pense à Dieu. Il marche sur scène, l'air serein.
	Et il prend enfin son téléphone.}
	\begin{itemize}
	\repItem{\henry}{Allô, Madame la directrice, j'ai pris ma décision.
		J'accepte de travailler chez vous et je reste à Paris.}
	\end{itemize}
	\dida{Le mentor revient sur scène. Il tend la main à Henry. Henry l'accepte :
		ils se réconcilient.}