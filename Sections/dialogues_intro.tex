\subsection{Introduction}

	\subsubsection{Recrutement}

	\dida{Les 2 recruteurs entrent, l'un à droite de la scène, 
	l'autre à gauche. Puis ils s'immobilisent.}	
	\begin{itemize}
	\repItem{\henry{} \dida{Voix-off}}{\off{OK, très bien. Voici comment
		tout a commencé.}}
	\end{itemize}	
	\dida{Les recruteurs prennent la parole tour à tour mais
	ils ne se voient pas. Ils parlent à leurs collègues, invisibles.}
	
	\dida{On commence dans l'association chrétienne.}
	\begin{itemize}
	\repItem{\assoResp}{Ensuite, c'est au tour d'Henry. Qu'est-ce que vous
	en pensez ? Yohann ?}
	\repItem{Yohann}{\ldots} \dida{Le responsable acquiesce de la tête.}
	\repItem{\assoResp}{Oui, tu as raison. C'est quelqu'un de bien.
		Sa foi est ferme, il n'a y aucun doute. 
		\dida{Silence}
		Par contre, j'ai un doute sur, comment dire\ldots} 
	\repItem{Marion}{\ldots}
	\repItem{\assoResp}{Oui, c'est ça, Marion. J'ai peur qu'il ne soit
		pas assez proche des autres membres du groupe.}
	\end{itemize}
	
	\dida{On bascule dans l'entreprise.}
	\begin{itemize}
	\repItem{\compDir}{Le prochain candidat est : Henry \nom{Charles}.
		La parole est à vous.} \dida{Il regarde et écoute ses collaborateurs.}
	\repItem{Collaborateurs}{\ldots}
	\repItem{\compDir}{Je suis d'accord avec vous : M. \nom{Charles} est 
		quelqu'un de compétent. Mais il aura énormément de mal à s'intégrer
		dans l'équipe.}
	\repItem{Un collaborateur}{\ldots}
	\repItem{\compDir}{Non, je ne juge pas sur l'apparence. J'ai pu voir ce 
		qu'il valait pendant son stage ici, dans notre entreprise.}
	\end{itemize}
	
	\dida{On retourne à l'association.}
	\begin{itemize}
	\repItem{\assoResp}{Bon, les amis, on doit prendre une décision.
		Est-ce que quelqu'un est contre ?
		Oui, Jonathan. Qu'est-ce qu'il y a ?}
	\repItem{Jonathan}{\ldots}
	\repItem{\assoResp}{Tu as raison de soulever ce point.
		\dida{Il s'adresse à tous.} En plus de cette candidature pour le
		voyage missionnaire au Mali, Henry a aussi passé un entretient pour
		travailler dans une entreprise informatique\ldots \dida{Quelqu'un
		l'interrompt.}
		Oui, Paul ?}
	\repItem{Paul}{\ldots}
	\repItem{\assoResp}{Ce que tu dis est très juste. C'est assez étrange
		d'envisager ses deux directions plutôt opposées. Mais je pense qu'il
		cherche	simplement la volonté de Dieu.}
	\end{itemize}
	\dida{Le responsable sort de la scène.}
	
	\dida{Dernier switch dans l'entreprise.}
	\begin{itemize}
	\repItem{\compDir}{Chers collègues, nous devons nous décider rapidement.
		Nous avons d'autres candidatures à analyser.} % autre mot ?
	\repItem{Collaborateurs}{\dida{Ils discutent. La directrice les regardent
		et écoutent.}}
	\repItem{\compDir}{\dida{Elle s'impatiente.} Excusez-moi, mais je crois que
		nous n'avons plus le temps. Je vous propose de revenir sur ce dossier 
		à la fin.}
	\end{itemize}
	\dida{La directrice sort de la scène.}
		
	\subsubsection{Le choix}
	
	\dida{Henry entre en scène avec son kit main libre et 2 lettres.}
	\begin{itemize}
	\repItem{\henry}{Allô maman ? C'est moi, Henry.}
	\repItem{\maman}{\ldots}
	\repItem{\henry}{Oui, maman. J'ai reçu les réponses, sous formes
		de lettres.}
	\repItem{\maman}{\ldots}
	\repItem{\henry}{Ce qu'elles disent ? 
		\dida{Il tend une première lettre d'un côté.}
		La première lettre me dit que ma candidature pour le voyage
		missionnaire a été retenue. 
		\dida{Il tend l'autre lettre.}
		Et la deuxième lettre m'informe que l'entreprise\ldots
		a accepté de m'embaucher.}
	\repItem{\maman}{\dida{Elle crie de joie. Henry a mal aux oreilles.}}
	\repItem{\henry}{Oui, maman, qu'est-ce qu'il y a?}
	\repItem{\maman}{\ldots}
	\repItem{\henry}{Non, maman, je ne peux faire les deux à la fois.}
	\repItem{\maman}{\ldots}
	\repItem{\henry}{Pourquoi ? Bah, parce que je ne peux être au Mali
		et à Paris en même temps.}
	\repItem{\maman}{\ldots}
	\repItem{\henry}{Quelle décision je vais prendre ? Je ne sais pas.
		J'ai une semaine pour me décider.}
	\end{itemize}
	\dida{Henry sort de la scène.}