\section{Préparation}

\begin{frame}
\sectionFrame{}
\end{frame}

	\subsection{Synopsis et thèmes}

	\begin{frame}
	\frametitle{Synopsis}
	
		\begin{block}{Le pitch}
		Henry est fraîchement diplômé en informatique.
		Il hésite entre un premier emploi dans une entreprise
		et un voyage missionnaire avec une association
		chrétienne.
		\end{block}
		
		\begin{itemize}
		\item Reprise des éléments biographiques
		\item Suspense : quel choix ?
		\item Voyage dans le temps :
            \begin{block}{Histoireception}
            Henry raconte à ses collègues son parcours personnel
            et les motivations de son choix.
            \end{block}
		\end{itemize}
		
	\end{frame}
	
	\begin{frame}
	\frametitle{Thèmes}
	
		\begin{block}{\textbf{Thème principal}}
		{\bfseries Oser parler de Dieu et de l'Evangile à ses collègues.}
		\end{block}
		
		\begin{block}{Thèmes secondaires}
			\begin{itemize}
			\item Critique d'un monde qui court à sa perte : recherche du 
			profit et des plaisirs au détriment du long terme et du durable.
			\item Introduction au monde informatique (et geek).
			\item De l'isolement d'Henry à sa libération.
			\end{itemize}
		\end{block}
		
	\end{frame}

	\subsection{Personnages}
	
	\begin{frame}
	\frametitle{\insertsubsection}
	
		\begin{itemize}
		\item \perso{Henry} : un geek asocial - \cast{Salomon}
		\item \perso{2 recruteurs} : 
			\begin{itemize}
			\item Responsable de l'association : encourageante  -  
				\cast{Placidia}
			\item Directrice de l'entreprise : directe - \cast{Elisabeth}
			\end{itemize}
		\item \perso{4 proches} qui le conseillent :
			\begin{itemize}
			\item Sa maman - \cast{Margaret}
			\item Un ami - \cast{Billy}
			\item La fille dont il est amoureux - \cast{Arlette}
			\item Son meilleur ami et mentor - \cast{Davy}
			\end{itemize}
		\item \perso{2 collègues} de travail :
			\begin{itemize}
			\item La collègue qui s'en fout - \cast{Joséphine}
			\item Le collègue qui s'intéresse - \cast{David}
			\end{itemize}
		\end{itemize}
		
	\end{frame}
	
	\subsection{Environnement}
	
	\begin{frame}
	\frametitle{\insertsubsection}
	
		\begin{block}{L'idée}
		Le monde dans lequel nous vivons,
		les gens que nous côtoyons.
		\end{block}
		
		\begin{itemize}
		\item En région parisienne
		\item Monde urbain, \ofg{élitiste}, e.g. La Défense
		\item Athées, agnostiques
		\end{itemize}
		
	\end{frame}
	
	\begin{frame}
	\frametitle{\insertsubsection}
	
		\begin{figure}
		\centering
		\includemedia[
			width=0.6\linewidth,height=0.45\linewidth,
			activate=pageopen,
			flashvars={
				modestbranding=1 % no YT logo in control bar
				&autohide=1 % controlbar autohide
				&showinfo=0 % no title and other info before start
				&rel=0 % no related videos after end
			},
			url % Flash loaded from URL
			]{}{http://www.youtube.com/v/nvv9acA5cgo?rel=0&autoplay=1}
		\caption{La Défense - vue aérienne}
		\end{figure}
	
	\end{frame}
	
	\subsection{Mise en scène}

	\begin{frame}
	\frametitle{\insertsubsection}
	
		\begin{block}{Jeux d'oppositions}
			\begin{itemize}
			\item Imaginaire / Réaliste
			\item Symbolique / Flou
			\item Flashback / Live
			\end{itemize}
		\end{block}
		
		\begin{enumerate}
		\item Dans la première partie, le public doit se rendre compte que 
			quelque chose cloche.
		\item La seconde partie doit être réaliste pour marquer la différence.
		\end{enumerate}
	
	\end{frame}