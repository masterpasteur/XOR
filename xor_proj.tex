\documentclass{article}

\input sketch.sty

\def\henry{Henry}
\def\assoResp{Responsable de l'association}
\def\compDir{Directeur de l'entreprise}
\def\maman{Maman}
\def\ami{Ami}
\def\fille{Belle fille}
\def\mentor{Mentor}
\def\collOut{Collègue out}
\def\collIn{Collègue in}

\usetheme{Warsaw}
%\usecolortheme{crane}
\useinnertheme[shadow=true]{rounded}
\useoutertheme{shadow}
\mode<presentation>
\setbeamertemplate{navigation symbols}{}
\setbeamercovered{transparent}

\title[XOR]{Sketch de la fête de Noël 2012 :\\ XOR (eXclusive OR)}
\author[EEL.GdJ]{Groupe de jeunes de l'Eglise Evangélique du Lac}
\date{Séance 3 : 24 novembre 2012}

\begin{document}

\begin{frame}
\titlepage
\end{frame}

\begin{frame}
\textit{A Nicolas}
\end{frame}

\begin{frame}
\frametitle{Présentation}
\begin{abstract}
	\begin{itemize}
	\item Cette année, je devais choisir entre un stage avec les GBU
	(Groupes Bibliques Universitaires) et un doctorat de physique.
	\item Ce sketch s'inspire de ce choix difficile.
	\item Son message est simple : d'abord évangéliser nos proches,
	nos amis et nos collègues.
	\end{itemize}
\end{abstract}
\end{frame}

\begin{frame}[allowframebreaks]
\frametitle{Table des matières}
\setcounter{tocdepth}{2}
\tableofcontents
\end{frame}

\input Sections/prep.tex
	
\input Sections/scena.tex

\section{Dialogues}

\begin{frame}
\sectionFrame{}
\end{frame}

	\input Sections/dialogues_intro.tex
	\input Sections/dialogues_disc.tex
	\input Sections/dialogues_evang.tex

\input Sections/toDo.tex

\input Sections/insp.tex
	
\end{document}